\chapter{GL-117 Installation}
\label{chap:installation}

This chapter describes the requirements of \emph{GL-117} and its installation
concerning esp. the libraries required to compile and execute the game.

\section{Requirements}
\label{sec:requirements}

\emph{GL-117} requires Linux/Unix or MSWindows as operating system as well as
properly installed versions of the following libraries:
\begin{itemize}
\item{OpenGL or MesaGL: graphics library, 3D engine}
\item{GLU or MesaGLU: utilities for GL}
\item{GLUT or MesaGLUT: a toolkit that provides keyboard and mouse support}
\item{SDL (optional): the Direct Media Layer library has similar features of GLUT plus joystick
support and basic sound processing}
\item{SDL\_mixer (optional): a library that provides advanced multichannel sound
support and music}
\end{itemize}

Installation of \texttt{SDL} is optional, however strongly recommended.

\section{Downloading GL-117}
\label{sec:downloading_gl117}

The latest \emph{GL-117} release is currently available for download at
\texttt{http://www.freshmeat.net/gl-117} in \texttt{.tar.gz} and \texttt{.zip} format.
Using MSWindows you may prefer the \texttt{.zip} file that may be unpacked with
lots of programs like \texttt{PKUnzip, Winzip, WinRAR, WinACE}.
The \texttt{.tar.gz} version can be unpacked with GNU Tar using \\\texttt{tar xvfz
gl-117-x.y.z.tar.gz}\\ where \texttt{x.y.z} is the \emph{GL-117} version
number. For last minute updates and release-specific building and
install instructions, make sure to have a look at the
\texttt{README} and \texttt{INSTALL} files.

\section{Linux/Unix installation}
\label{sec:linux_installation}

If you got a binary \texttt{gl-117} in the \texttt{linux} directory, you will only
need the libraries \texttt{GL, GLU, GLUT} and \texttt{SDL, SDL\_mixer} as
described above.

In order to compile \emph{GL-117} you will also have to install the developement
versions of the libraries above (except \texttt{SDL\_mixer}).
To compile \emph{GL-117} do the following steps in the gl-117-\textit{VERSION}
directory:
\begin{verbatim}
 ./configure
 make
\end{verbatim}

The \texttt{configure} script will check for the required libraries and will output
a \texttt{Makefile}, which can be invoked using the \texttt{make} command.
After compiling \emph{GL-117} successfully, you will find a binary called \texttt{gl-117}
in the \texttt{src} directory. Move it to the \texttt{linux} directory manually and
execute.\\
To really install \emph{GL-117} please use:
\begin{verbatim}
 make install
\end{verbatim}
This will copy the binary to your binary directory (e.g. \texttt{/usr/local/bin})
and the rest of data files to your data directory (e.g. \texttt{/usr/local/share}).
Any files that require output permissions will be stored in the user's home directory,
exactly \texttt{\$HOME/.gl-117}.\\
This step will require write permissions in the binary and data directories.
However you may customize these directories using for example
\begin{verbatim}
 ./configure --prefix='/home/tom/gl-117'
 make
 make install
\end{verbatim}


\section{MSWindows installation}
\label{sec:windows_installation}

First, you might have to install \texttt{GL, GLU, GLUT} and \texttt{SDL, SDL\_mixer}.
Look into your system directory, that is generally
\begin{verbatim}
 \WINDOWS\SYSTEM    for Windows9x/ME
 \WINDOWS\SYSTEM32  for WindowsNT/2000/XP
\end{verbatim}
You will need the files \texttt{opengl32.dll, glu32.dll, glut32.dll, sdl.dll, sdl\_mixer.dll}
there.
If one is missing, please search the internet.
That's it. Execute the binary \texttt{gl-117.exe} in \emph{GL-117}'s \texttt{windows}
directory.\\

If you had already an earlier version of \emph{GL-117} you might want to use your
old pilots with the new version of the game. Therefore simply copy the old
\texttt{saves} directory to the new version.


\section{Running GL-117}
\label{sec:running_gl117}

At startup, \emph{GL-117} tries to read a file \texttt{conf} from the user's home
directory (Linux/Unix) or the \texttt{saves} directory (MSWindows).
If there is none, the game will try out some screen settings and store the file.
\begin{verbatim}
 Linux/Unix      $HOME/.gl-117/conf
 MSWindows       GL-117-INSTALLDIR/saves/conf
\end{verbatim}

Edit the file using your favourite text editor and adjust the settings
to your system.
If you lack a hardware accelerated video card, please turn down the
quality to 0 or 1. Further acceleration can be achieved negligating fullscreen mode
and choosing a lower resolution.\\
Just delete the \texttt{conf} file if you want to reset to the initial settings.
